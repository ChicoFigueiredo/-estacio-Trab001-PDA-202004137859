\chapter{Procedimentos Metodológicos}

\section{Pré-requisitos}

Como instruído:
\begin{itemize}
    \item Desenvolver visão panorâmica da Psicologia do Desenvolvimento e da Aprendizagem por meio dos referenciais teóricos estudados nas aulas.
    \item Entender as relações entre educação, desenvolvimento biológico e aprendizagem.
    \item Capacidade de observação e articulação entre teoria e prática.
    \item Observar e coletar dados  - você irá observar uma criança/adolescente que pode ter de 3 meses até 19 anos de idade, por alguns dias,  e estabelecer a relação entre os aspectos psicossocial, biossocial e cognitivo dessa criança. Essa criança/adolescente pode ser da sua família ou ser sua vizinha. Ou ainda pode observar uma criança em um dos estágios que realizou ou está realizando no curso.
    \item Estabelecer a relação entre o que está coletando e as leituras realizadas nas aulas.
    \item Ao menos 9h de observação (como descrito Frequência das observações – 3 dias de observações e Tempo de duração de cada observação – 3 horas por encontro)
\end{itemize}

\section{Pesquisa}

Dado o contexto, era necessário estabelecer critérios para seleção do objeto de estudo. Esses critérios visam preservar os pré-requisitos e não desviar do foco do objetivo desse trabalho, logo foram os seguintes critérios para poder escolher o objeto de estudo:

\begin{itemize}
    \item Escolher um canal no YouTube, pelo alcance e audiência
    \item O canal escolhido seja um adolescente de no máximo 19 anos de idade
    \item O canal escolhido ter conteúdo de ao menos 3 anos de vídeos frequentes (ao menos 1 por mês em média)
    \item O canal ter vídeos mais abrangentes e com interação com fãs e membros da família
\end{itemize}

\section{Reflexão}


Apresentação no OneDrive: \url{https://1drv.ms/p/s!AgRBucATAhUblzAldnG4LGnWNV-r?e=ykIvGi} \\
\begin{center}
    \href{https://1drv.ms/p/s!AgRBucATAhUblzAldnG4LGnWNV-r?e=ykIvGi}{
        \qrcode{https://1drv.ms/p/s!AgRBucATAhUblzAldnG4LGnWNV-r?e=ykIvGi}
    }
\end{center}


Vídeo no YouTube: \url{https://youtu.be/szsZ_Uuk1zk} \\
\begin{center}
    \href{https://youtu.be/szsZ_Uuk1zk}{
        \qrcode{https://youtu.be/szsZ_Uuk1zk}
    }
\end{center}
