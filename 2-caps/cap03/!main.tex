\chapter{Procedimentos Metodológicos}

\section{Pré-requisitos}

Como instruído:
\begin{itemize}
    \item Aplicar de modo reflexivo paradigmas e conceitos estudados na disciplina para a construção de práticas de educação, que tornem mais equilibradas as relações entre a sociedade e o meio ambiente.
    \item Observar as diferentes interações estabelecidas pelas comunidades com os recursos naturais do entorno sociopolítico cultural onde vive e convive.
    \item Aplicar na prática conceitos da Sociologia e da Antropologia relacionados às práticas de educação e desenvolvidos pela disciplina Aspectos Antropológicos e Sociológicos da Educação, adotando postura de futuro (a) professor (a).
\end{itemize}

\section{Pesquisa}

Dado o contexto, era necessário estabelecer critérios para seleção do objeto de estudo. Esses critérios visam preservar os pré-requisitos e não desviar do foco do objetivo desse trabalho, logo foram os seguintes critérios para poder escolher o objeto de estudo:

\begin{itemize}
    \item Escolher um canal no YouTube, pelo alcance e audiência
    \item O canal escolhido seja um adolescente de no máximo 19 anos de idade
    \item O canal escolhido ter conteúdo de ao menos 3 anos de vídeos frequentes (ao menos 1 por mês em média)
    \item O canal ter vídeos mais abrangentes e com interação com fãs e membros da família
\end{itemize}

\section{Reflexão}


Apresentação no OneDrive: \url{https://1drv.ms/p/s!AgRBucATAhUblzAldnG4LGnWNV-r?e=ykIvGi} \\
\begin{center}
    \href{https://1drv.ms/p/s!AgRBucATAhUblzAldnG4LGnWNV-r?e=ykIvGi}{
        \qrcode{https://1drv.ms/p/s!AgRBucATAhUblzAldnG4LGnWNV-r?e=ykIvGi}
    }
\end{center}


Vídeo no YouTube: \url{https://youtu.be/szsZ_Uuk1zk} \\
\begin{center}
    \href{https://youtu.be/szsZ_Uuk1zk}{
        \qrcode{https://youtu.be/szsZ_Uuk1zk}
    }
\end{center}
