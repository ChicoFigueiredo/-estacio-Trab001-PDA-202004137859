\chapter{Objetivos}

O presente trabalho visa ensaiar um estudo sociológico de Brasília sob a ótica da relação entre a sociedade e o meio ambiente, discutindo a importância da educação como fomentador para a conscientização e evolução do pensamento relacionado aos impactos ambientais causados pelo homem, em especial delimitado no quadrado geográfico incrustado no meio do planalto central.

Além de ser essencial para o desenvolvimento intelectual dos alunos a partir da reflexão sobre a importância da natureza, dos impactos diretos sobre a ação humana.

O trabalho será apresentado por meio de pesquisa direta em sites e vídeos na internet, por análises de documentos, de jornais e pesquisas realizadas por pesquisadores e alunos, sempre sob a perspectiva sociológica.

O local observado do trabalho será Brasília, o Distrito Federal, e suas regiões administrativas e os fatos serão o lixo como resultado direto da ação humana e educação ambiental. Portanto, terá a articulação entre a teoria e a prática , que consistirá na observação e identificação dos problemas sociais que afetam a qualidade de vida da população, e como o contraste comparativo presente em áreas da cidade e como isso passa despercebido pela sociedade.

Ademais, vale considerar que, pelo fato do autor pertencer aos grupos de risco, a pesquisa teve que ser feita remotamente, sem ir a campo, em virtude do risco de contaminação pelo vírus da COVID 19.


Esse documento foi programado em \LaTeX, MikTeX, abntex2 e todo conteúdo possui links referenciais clicáveis, sejam tabelas, figuras, imagens de vídeos, autores com seu respectivo registro bibliográfico.
Informamos que projeto gerador desse PDF está disponível no endereço (legível também pelo QR Code abaixo): \\
\url{https://github.com/ChicoFigueiredo/estacio-Trab001-AASE-202004137859.git} \\
\qrset{link, height=4cm}
\begin{center}
    \href{https://github.com/ChicoFigueiredo/estacio-Trab001-AASE-202004137859.git}{
        \qrcode{https://github.com/ChicoFigueiredo/estacio-Trab001-AASE-202004137859.git}
    }
\end{center}


Apresentação no OneDrive: \url{https://1drv.ms/p/s!AgRBucATAhUblzAldnG4LGnWNV-r?e=ykIvGi} \\
\begin{center}
    \href{https://1drv.ms/p/s!AgRBucATAhUblzAldnG4LGnWNV-r?e=ykIvGi}{
        \qrcode{https://1drv.ms/p/s!AgRBucATAhUblzAldnG4LGnWNV-r?e=ykIvGi}
    }
\end{center}




Vídeo no YouTube: \url{https://youtu.be/szsZ_Uuk1zk} \\
\begin{center}
    \href{https://youtu.be/szsZ_Uuk1zk}{
        \qrcode{https://youtu.be/szsZ_Uuk1zk}
    }
\end{center}
