\chapter{Objetivos}

O presente trabalho tem por objetivo observação de um jovem do ensino médio para compreender características  que ressaltam o  desenvolvimento biossocial, cognitivo e psicossocial.

Em virtude da pandemia do corona vírus (COVID-19) o objeto de observação será feito por meio de redes sociais, notadamente pelo YouTube, sem que se perca a perspectiva do trabalho em epígrafe.

Esse documento foi programado em \LaTeX, MikTeX, abntex2 e todo conteúdo possui links referenciais clicáveis, sejam tabelas, figuras, imagens de vídeos, autores com seu respectivo registro bibliográfico.

Informamos que projeto gerador desse PDF está disponível no endereço (legível também pelo QR Code abaixo): \\
\url{https://github.com/ChicoFigueiredo/-estacio-Trab001-PDA-202004137859.git} \\
\qrset{link, height=4cm}
\begin{center}
    \href{https://github.com/ChicoFigueiredo/-estacio-Trab001-PDA-202004137859.git}{
        \qrcode{https://github.com/ChicoFigueiredo/-estacio-Trab001-PDA-202004137859.git}
    }
\end{center}


Apresentação no OneDrive: \url{https://1drv.ms/p/s!AgRBucATAhUblzAldnG4LGnWNV-r?e=ykIvGi} \\
\begin{center}
    \href{https://1drv.ms/p/s!AgRBucATAhUblzAldnG4LGnWNV-r?e=ykIvGi}{
        \qrcode{https://1drv.ms/p/s!AgRBucATAhUblzAldnG4LGnWNV-r?e=ykIvGi}
    }
\end{center}


Vídeo aula no YouTube explicando o conteúdo: \url{https://youtu.be/szsZ_Uuk1zk} \\
\begin{center}
    \href{https://youtu.be/szsZ_Uuk1zk}{
        \qrcode{https://youtu.be/szsZ_Uuk1zk}
    }
\end{center}
