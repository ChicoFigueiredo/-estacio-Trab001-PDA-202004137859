\chapter{Resultados e Conclusão}

\section{Resultados}

Concluímos que a formulação da metodologia aplicada para coleta de informações se mostrou satisfatória e permitiu fazer uma análise psicossocial adequada do objeto de estudo.

O objeto de estudo se mostrou surpreendentemente rico em informações que permitiram aplicar as teorias estudadas na matéria além de permitir uma reflexão profunda tanto do adolescente estudado quanto de um subconjunto pequeno de estudantes periféricos à protagonista.

Os dados coletados e a reflexão subsequente permitiram conclusões e reflexões, embora sucintas, mas densas em conteúdo.

Convém sugerir também que o exposto no item \ref{Reflexão} referente à diminuição da Zona de Desenvolvimento Proximal por meio do YouTube seja objeto de uma pesquisa mais ampla.

\section{Conclusão}

O objeto adolescente é muito rico, o retrato aqui exposto é muito pequeno e muito rico que essa obra, devido a limitações de tempo e escopo, não consegui transmitir de forma que o tema merece.

Foi fácil compreender o exposto nos diversos aspectos da Psicologia do Desenvolvimento e da Aprendizagem por meio de visualização prática de um objeto de pesquisa tão fascinante quanto a Zabetta.



