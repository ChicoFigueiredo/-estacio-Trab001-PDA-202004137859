\section{Da escolha do objeto e ambiente de estudo}

Como cita \citeaa{Sousa2017} em seu ensaio: "Diante das transformações ocorridas na sociedade contemporânea, com o advento da internet vivemos em uma sociedade conectada com as mídias sociais e com isso um cidadão comum pode se tornar mais conhecido e influenciar o comportamento de jovens. Isso pode ser possível com uma câmera na mão, uma ideia na cabeça e uma plataforma como, por exemplo, o You Tube, então surge o fenômeno conhecido como youtuber. Essas pessoas são fenômenos midiáticos, pois ao gravar vídeos sobre assuntos variados para entreter seu público, conseguem a adesão de milhares de pessoas e acabam lucrando/ganhando em média US\$ 1 a cada mil visualizações por vídeos."

E, em complemento, os ensaistas relatam que "Os resultados obtidos foram que as temáticas preferidas pelos adolescentes foram: Humor, imitações, seriados, jogos online, beleza, trolagens\footnote{\textbf{Trolar} é uma gíria da internet que significa zoar, chatear, tirar o sarro, fazer uma pegadinha. Consiste em sacanear os participantes de uma discussão em fóruns da internet, com argumentos sem sentido, apenas para enfurecer e perturbar a conversa. Atualmente, o ato de trolar alguém não acontece só no ambiente virtual. Trolar vem do inglês \href{https://www.oxfordlearnersdictionaries.com/definition/english/troll_2}{troll} }, orientação sexual, questões sociais, políticas, estudos, futebol e músicas. Os participantes afirmam ficar em média 5 horas por dia conectados assistindo vídeos e seguindo os youtubers nas redes sociais para saberem o assunto do próximo vídeo, bem como, acompanhar a vida dos mesmos. Todos participantes afirmaram que se inspiram em pelo menos em um youtuber para tomar uma decisão ou ter opinião sobre algum fato do cotidiano. Portanto, os resultados obtidos foram satisfatórios aos nossos objetivos, pois os adolescentes através da observação tendem a imitar alguns comportamentos que os youtubers apresentam. Reforço vicariante, modelagem, desinibição e autoeficácia foram alguns dos conceitos que identificamos no processo de modelagem no comportamento e personalidade dos adolescentes, isso pela influência dos youtuberes. Inferimos que adolescentes na busca de si e de sua identidade, procuram no youtuber inspiração para modelar seu comportamento e sua personalidade, tomando para si, as características dos modelos. As características dos observadores e as consequências recompensadoras associadas ao comportamento, reforçam ou enfraquecem os comportamentos já existentes nos adolescentes."

Observar um adolescente no YouTube parece ser uma excelente oportunidade de analisar sobre os aspectos solicitados nesse relatório como também ter consciência do que é tratado pelas crianças e adolescentes na relação ao conteúdo apresentado e tentar estabelecer vínculo com os tópicos apresentados.

\section{Da observação e linguagem}

\citeaa{Costa2009} cita que "As videografias de si constituem-se como pequenas autobiografias em vídeo do YouTube, nas quais a enunciação de si contém de fortes tons midiáticos. Nelas, são descritas e narradas experiências do cotidiano, impressões e análises de si, geralmente ancoradas em situações corriqueiras do dia a dia. Elas são produtos de indivíduos para os quais o registro e a exibição de si em vídeo se torna tanto um modo de representação como uma expressão de subjetividade. Nas videografias, essa dupla função se articula com um viés confessional para constituir sua especificidade."

E complementa: "Por um lado, as práticas de representação emulam a linguagem e a lógica dos meios de comunicação massivos. Olhadas a partir da perspectiva teórica de Luhmann (1990, 2005)\footnote{“Here I am assuming an information concept where by something can be regarded as information only if it is selected according to the criterion of difference.” (LUHMANN, 1990, p. 98)}, essa consonância se configuraria graças ao contato dos indivíduos com as práticas de representação dos meios de comunicação. Desse modo, pode-se considerar que os indivíduos replicam em seus próprios produtos o mesmo critério de diferenciação que esses meios usam para determinar o que é ou não informação."