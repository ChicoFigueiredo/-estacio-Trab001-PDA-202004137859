\section{Análise}

Iremos separar a análise sob os aspectos abordados em nosso curso:

\subsection{Aspectos Físico-Motor}

Segundo \citeaa{Carraro2015}, aspectos físico-motores são referentes ao crescimento orgânico, à maturação neurofisiológica, à capacidade de manipulação de objetos e de exercício do próprio corpo.

\subsection{Aspectos Intelectuais e Cognitivos}

Segundo \citeaa{Carraro2015}, "fazem parte do desenvolvimento cognitivo dos indivíduos as questões relativas ao desenvolvimento dos comportamentos inteligentes: a memória, a linguagem e a comunicação; ao aprendizado de leis e lógica; aos números, padrões e formas; aos potenciais e habilidades, o processamento das habilidades sensoriais. Para ser considerado inteligente, o comportamento deve ser orientado para a meta, isto é, consciente e não acidental, e adaptativo em relação às circunstâncias.

Ainda segundo o material on-line do curso, os processos cognitivos respeito aos processos mentais utilizados para obtermos conhecimentos ou para nos tornarmos conscientes do ambiente. A cognição abrange percepção, imaginação, discernimento, memória e linguagem, que são processos utilizados pelas pessoas para pensar, decidir e aprender.  A educação, incluindo o currículo formal das escolas, a instrução informal proporcionada pela família e pelos amigos, e outros espaços sociais, além do resultado da curiosidade e da criatividade individuais, também fazem parte desse domínio.
Em nosso estudo, utilizaremos os conhecimentos construídos pela ciência, por pesquisadores do desenvolvimento e da aprendizagem.

A ciência se utiliza do método científico que parte da: observação da realidade, identificando um problema de estudo; elaboração de hipóteses; testagem das hipóteses; e elaboração de conclusões com base nesses dados. A etapa final de uma pesquisa científica deve culminar na socialização e/ou publicação de seu processo e resultados.




\subsection{Aspectos Biossociais}

Os aspectos biossociais dizem respeito ao crescimento e às modificações que ocorrem no corpo de uma pessoa, além dos fatores genéticos, nutricionais e de saúde que afetam esse crescimento e tais modificações. Também fazem parte do desenvolvimento biossocial: as habilidades motoras; os fatores sociais e culturais que afetam essas áreas como, por exemplo, a duração da amamentação ao seio, a educação de crianças com necessidades especiais e as atitudes quanto às formas ideais do corpo.


\subsection{Aspectos Psicossociais}

Diz respeito ao desenvolvimento das emoções, do temperamento e das habilidades sociais. As influências da família, dos amigos, da comunidade, da cultura e da sociedade como um todo são especialmente importantes para esse domínio. Assim sendo, diferenças culturais relativas ao valor das crianças ou às ideias sobre os papéis “apropriados” a cada sexo, ou ao que é considerado a estrutura familiar ideal, são considerados parte desse domínio.

