\section{Reflexão}

Após a observação, eu me atrevo a estabelecer uma ligação entre o conceito de Zona de Desenvolvimento Proximal ao poder que o Youtube tem.

Como proposto por Vygotsky, maravilhosamente explicado em \citeaa{Ivic2010} temos que '
A noção vygotskyana de “zona de desenvolvimento proximal” tem, de início, uma marca teórica. Na concepção sociocultural de desenvolvimento, a criança não deveria ser considerada isolada de seu contexto sociocultural, em uma espécie de modelo RobinsonCrusoé-criança. Seus vínculos com os outros fazem parte de sua própria natureza. Desse modo, nem o desenvolvimento da criança, nem o diagnóstico de suas aptidões, nem sua educação podem ser analisados se seus vínculos sociais forem ignorados. A noção de zona de desenvolvimento proximal ilustra, precisamente, esta concepção. Esta zona é definida como a diferença (expressa em unidades de tempo) entre os desempenhos da criança por si própria e os desempenhos da mesma criança trabalhando em colaboração e com a assistência de um adulto. Por exemplo, duas crianças têm sucesso nos testes de uma escala psicométrica correspondente à idade de 8 anos; mas, com uma ajuda estandartizada, a primeira não alcança senão o nível de 9 anos, enquanto a segunda atinge o nível de 12; enquanto a zona proximal da primeira é de um ano a da outra é de quatro anos. '

E mais adiante, 'Nessa noção de zona proximal, a tese da criança como ser social gera um aporte metodológico de grande significado, uma vez que ele enfoca o desenvolvimento da criança no seu aspecto dinâmico e dialético. Aplicada à pedagogia, essa noção permite sair do eterno dilema da educação: é necessário esperar que a criança atinja um nível de desenvolvimento particular para começar a educação escolar, ou é necessário submetê-la a uma determinada educação para que ela atinja tal nível de desenvolvimento? Na linha das ideias dialéticas das relações entre processos de aprendizagem e de desenvolvimento que analisamos, Vygotsky acrescenta que este último é mais produtivo se a criança é exposta a aprendizagens novas, justamente na zona de desenvolvimento proximal. Nessa zona, e em colaboração com o adulto, a criança poderá facilmente adquirir o que não seria capaz de fazer se fosse deixada a si mesma.'

A ligação é que a internet e, mais especificamente o YouTube, está reduzindo dramaticamente a distância entre as pessoas e, por conseguinte, a Zona de Desenvolvimento Proximal. Vejamos, são raros os casos que não exista um tutorial no YouTube de como fazer. Apertar o parafuso específico de um carro,