\iffalse
Roteiro

1 – Objetivos: Mencionar de forma clara e resumida quais são os principais objetivos do Relatório.

2 – Introdução Teórica: Este item deverá conter informações teóricas ou históricas importantes sobre o assunto abordado, com conteúdo escrito de forma direta, preferencialmente com períodos curtos de construção gramatical e obedecendo as regras tradicionais da escrita. Utilizar autores das aulas e outros trazidos de novas leituras.

3 – Procedimentos Metodológicos: Descrever de forma dissertativa quais foram os materiais e métodos empregados na aula prática para a sua execução.

4 – Resultados e Conclusão: Apresentar os resultados obtidos na observação e a relação estabelecida entre os procedimentos metodológicos e a discussão teórica.

5 – Referências Bibliográficas: Colocar as obras literárias utilizadas para a preparação do Relatório tais como: livros, normas técnicas, artigos de revistas ou outras obras, sempre respeitando as normas padrões para elaboração das referências bibliográficas da ABNT.






	Desenvolver visão panorâmica da Psicologia do Desenvolvimento e da Aprendizagem por meio dos referenciais teóricos estudados nas aulas.
	Entender as relações entre educação, desenvolvimento biológico e aprendizagem.
	Capacidade de observação e articulação entre teoria e prática.

	Observar e coletar dados  - você irá observar uma criança que pode ter de 3 meses até 12 anos de idade, por alguns dias,  e estabelecer a relação entre os aspectos psicossocial, biossocial e cognitivo dessa criança. Essa criança pode ser da sua família ou ser sua vizinha. Ou ainda pode observar uma criança em um dos estágios que realizou ou está realizando no curso.
	Estabelecer a relação entre o que está coletando e as leituras realizadas nas aulas.
	Escrita do relatório -  escrever um texto acadêmico, com referências bibliográficas e Ilustrar esse texto com exemplos advindos da observação realizada na criança. Não deixe de fazer a relação com as referências vistas nas aulas.

Local de observação – Local em que possa ter acesso à observação do público-alvo, se for possível. Caso não seja, considerar o artigo indicado e instruções já passadas.
Método de pesquisa – observação sistemática.
Frequência das observações – 3 dias de observações.

Tempo de duração de cada observação – 3 horas por encontro.

deverá escolher uma pessoa da faixa etária indicada: bebês (0 a 2 anos); crianças (2 a 11 anos); adolescentes (12 a 19 anos). Lembre-se que é apenas um sujeito (não precisa ser um de cada faixa etária, apenas um sujeito para todo o trabalho. Caso você não tenha uma criança em casa ou próxima de você, opte por ler o artigo que está disponível em http://www.scielo.br/pdf/paideia/v21n50/16.pdf  . Essa opção é para contornamos o período de isolamento social que estamos atravessando nos próximos meses. O relatório deverá ser elaborado com base nas informações de comportamentos descritas sobre as crianças observadas no artigo.
Em seguida, você deverá fazer uma observação sistemática do indivíduo em relação a 5 áreas: cognitiva, psicomotora, emocional, moral e social. ?Para isso, leve folha de papel e caneta para anotar todos os aspectos possíveis destas áreas. Para entender mais sobre elas, você deverá ter assistido e estudado, pelo menos, até a aula 5 da disciplina "Psicologia do Desenvolvimento e Aprendizagem".
Sua postura deverá ser apenas de observador. Você não deve ter qualquer interação com o indivíduo observado. Suas observações deverão ocorrer em 3 dias (não precisam ser consecutivos), sendo 3 horas em cada um deles.
Após fazer toda a observação, redija um relatório (conforme modelo ao final desta mensagem e conforme diretrizes postadas para o trabalho). Coloque seu nome, nome da disciplina, título do trabalho, e escreva o que observou em cada dia referente a cada área de desenvolvimento. Não coloque qualquer identificação do sujeito observado, caso você faça uma observação direta, ok?
A data de entrega máxima é 09/06/2020. O trabalho é obrigatório para aprovação na disciplina. Ele não vale nota, mas vale como atividade extra (que compõe a carga horária).
É forma de aplicar a parte teórica da disciplina a uma questão prática, apenas por meio de observação das áreas apontadas ou análise do artigo apresentado como alternativa.
Bons estudos!



\fi